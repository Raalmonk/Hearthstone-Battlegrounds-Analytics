# Hearthstone-Battlegrounds-Analytics
\documentclass{article}
\usepackage{amsmath,bm}

\documentclass{article}
\usepackage{amsmath,bm}

% Document metadata
\title{Hearthstone Battlegrounds Analytics}
\author{Raalm}
\date{\Jul 3 2025}

\begin{document}
\maketitle

\begin{abstract}
A data-driven framework for macro and micro analysis of Battlegrounds matches.
\end{abstract}

\section{Introduction}
目标:
宏观(对所有玩家):研究对局整体的的行为
微观(对单个玩家):MMR与个人能力的关系,

背景:
酒馆战棋Solo。

获胜条件:8players操控八个英雄。每个英雄都有30点生命值加上5-19点护甲(随英雄变化)。在每一个回合中,玩家会被分成四组1v1。获胜方会对输的一方造成酒馆等级+Σ(随从等级)的伤害。伤害有上限:The damage cap will start at 5 damage, then go up to 10 damage on turn 4, and 15 damage on turn 8. The cap will still be removed once you’re in the Top 4.想增加MMR,必须进入前四名。

Minion:所有玩家共用一个Minion池,即时刷新。每一张Minion的数量都有限,具体来说tier 1 的有15张,tier 2 15张,tier 3 13张, tier4 11张,tier 5 9张, tier 6 7张。

玩家的能力评定:
a0:宏观节奏:英雄选择、升本曲线、经济节奏。血量-经济平衡。
a1:商店交互:和商店交互的能力。做出完美决定的正确率。
a2:随从交互:能针对环境让随从发挥出百分之多少实力的能力。


\end{document}
